\chapter{矩量法}

\section{矩量法概述}

\par Maxwell 方程的解可以通过不同的方式得到。
除偏微分方程之外,也可以求解从 Maxwell 方程出发导出的积分
方程。

\par 例如,对于静电场问题,可以用 Green 函数法推导积分方程。
如果希望求解一块金属导体的电容,可以先建立一个积分方程来求解
导体表面的电荷。使用 Green 函数法,由线性叠加原理,可以写出
由金属导体表面电荷产生的总电势,即
\begin{equation}
    \varphi(\vec{r}) = 
    \iint_{S} G(\vec{r}, \vec{r}') \varrho_s(\vec{r}') \text{d}S'
    =\iint_{S} \frac{1}{4\pi\epsilon\left|\vec{r}-\vec{r}'\right|} 
    \varrho_s(\vec{r}') \text{d}S'
\end{equation}
\par 其中,$S$ 为金属体表面,
$\varrho_s(\vec{r}')$ 为金属表面的面电荷密度,
$\varphi(\vec{r})$ 为由面电荷产生的电势。
\par 在金属表面 $S$ 上,$\varphi(\vec{r})$
必须为常数,记为 $\Phi$,由此有
\begin{equation}
    \Phi = 
    \iint_{S} G(\vec{r}, \vec{r}') \varrho_s(\vec{r}') \text{d}S'
    \qquad
    \vec{r} \in S
    \label{金属表面积分方程}
\end{equation}
\par 上式就是一个积分方程,式中 $\Phi$ 为已知常数,
$G(\vec{r}, \vec{r}')$ 为已知函数,而 $\varrho_s(\vec{r}')$ 则为未知的金
属表面电荷密度,通过求解此积分方程可以得到 $\varrho_s(\vec{r}')$。
\par 为了使式 (\ref{金属表面积分方程}) 能使用计算机求解,可以用矩
量法在有限维子空间中对其解进行近似。首先,选择一组基函数用于
对 $\varrho_s(\vec{r}')$ 的近似,即
\begin{equation}
    \varrho_s(\vec{r}') = \sum_{j=1}^{N} c_j v_j(\vec{r}')
    \label{矩量法基函数展开}
\end{equation}
\par 其中,$v_j(\vec{r}')$ 为基函数,$c_j$ 为待定系数。
\par 选定基函数后,将式 (\ref{矩量法基函数展开}) 代人式 (\ref{金属表面积分方程}),
得到一个有限自由度的方程
\begin{equation}
    \sum_{j=1}^{N} c_j 
    \iint_{S} G(\vec{r}, \vec{r}') v_j(\vec{r}') \text{d}S'
    =\Phi
    \qquad
    \vec{r} \in S
    \label{矩量法方程}
\end{equation}
\par 为了求解 $c_j$,需要把上式转换为矩阵方程。
为此,选择一组函数 $w_1(\vec{r}),w_2(\vec{r}),\cdots,w_N(\vec{r})$,在
式 (\ref{矩量法方程}) 等号两边相乘,然后在整个 $S$ 面上进行积分,得到
\begin{equation}
    \sum_{j=1}^{N} c_j 
    \iint_{S} w_i(\vec{r})
    \iint_{S} G(\vec{r}, \vec{r}') v_j(\vec{r}') \text{d}S' \text{d}S
    =
    \iint_{S} w_i(\vec{r})
    \Phi \text{d}S
    \qquad
    i=1,2,\cdots,N
    \label{矩量法矩阵方程}
\end{equation}
\par 此方程可写成更紧凑的形式
\begin{equation}
    \sum_{j=1}^{N}
    A_{ij} c_j = b_i
    \qquad
    i=1,2,\cdots,N
\end{equation}

\begin{definition}{脉冲基函数}
    若划分的单元足够小,则每个面元上的电荷密度 $\varrho_s(\vec{r}')$ 可以认为是常数。这相当于
    使用下面的公式定义的零阶基函数
    \begin{equation}
        v_j(\vec{r}')=
        \left\{
            \begin{aligned}
                &1 \qquad \vec{r}' \in S_j \\
                &0 \qquad \vec{r}' \notin S_j
            \end{aligned}
        \right.
    \end{equation}
    其中,$S_j$ 为第 $j$ 个面元。
\end{definition}

\begin{definition}{点配置}
    需要选择一组检验函数 $w_i(\vec{r})$,最简单的选择是 $\delta$ 函数
    \begin{equation}
        w_i(\vec{r})=\delta(\vec{r}-\vec{r}_i)
    \end{equation}
    其中,$\vec{r}_i$ 为第 $i$ 个面元的中心。
\end{definition}

\begin{definition}{子域配置}
    除了 $\delta$ 函数,也可以选择零阶函数作为检验函数
    \begin{equation}
        w_i(\vec{r})=
        \left\{
            \begin{aligned}
                &1 \qquad \vec{r} \in S_i \\
                &0 \qquad \vec{r} \notin S_i
            \end{aligned}
        \right.
    \end{equation}
\end{definition}

\begin{theorem}
    对于式 (\ref{矩量法矩阵方程}) 给出的方程,
    若选择脉冲基函数和点配置,矩阵元素 $A_{ij}$ 和 $b_i$ 可以表示为
    \begin{align}
        \label{矩量法矩阵元素-1}
        A_{ij} &= \iint_{S_j} G(\vec{r}_i, \vec{r}') \text{d}S'\\
        b_i &= \Phi
    \end{align}
    式 (\ref{矩量法矩阵元素-1}) 的积分项可以用中点法进行数值积分计算,
    计算得到结果为
    \begin{equation}
        A_{ij} = 
        \left\{
            \begin{aligned}
                &\frac{1}{4\pi\epsilon}\frac{S_j}{\left|
                    \vec{r}_i-\vec{r}_j
                \right|} \qquad i \neq j \\
                &\frac{1}{2\epsilon}
                \sqrt{\frac{S_j}{\pi}} \qquad i = j
            \end{aligned}
        \right.
    \end{equation}
    其中,$\left|
        \vec{r}_i-\vec{r}_j
    \right|=\sqrt{(x_i-x_j)^2+(y_i-y_j)^2+(z_i-z_j)^2}$。
\end{theorem}

\begin{theorem}
    对于式 (\ref{矩量法矩阵方程}) 给出的方程,
    若选择脉冲基函数和子域配置,矩阵元素 $A_{ij}$ 和 $b_i$ 可以表示为
    \begin{align}
        A_{ij} &= 
        \iint_{S_i}
        \iint_{S_j} G(\vec{r}, \vec{r}') 
        \text{d}S' \text{d}S\\
        b_i &= \Phi S_i
    \end{align}
\end{theorem}

\par 求得 $c_j$ 之后,由于使用的是脉冲基函数,金属体上的总电荷可以用下式计算
\begin{equation}
    Q = \iint_{S} \varrho_s(\vec{r}') \text{d}S'
    =\sum_{j=1}^{N} c_j S_j
\end{equation}

\section{二维分析}

\subsection{积分方程的建立}

\par 考虑自由空间中的源 $f(\vec{\rho})$ 产生的标量波,而在此空间中有一个形状任
意的散射体。假设源和物体沿 $z$ 轴无变化,因此只需考虑任意一个垂直于 $z$ 轴的平面。
在散射体外部,波函数 $\varphi(\vec{\rho})$ 满足非齐次 Helmholtz 方程
\begin{equation}
    \nabla^2 \varphi(\vec{\rho}) + k_0^2 \varphi(\vec{\rho}) = f(\vec{\rho})
    \qquad
    \vec{\rho} \in \Omega_{\infty}
    \label{非齐次 Helmholtz 方程}
\end{equation}
\par 其中,$k_0$ 为波数,$\Omega_{\infty}$ 为散射体外部的区域。
波函数也应满足辐射条件
\begin{equation}
    \sqrt{\rho} \left[
        \frac{\partial \varphi(\vec{\rho})}{\partial \rho}
        -j k_0 \varphi(\vec{\rho})
    \right] = 0
    \qquad
    \rho \rightarrow \infty
    \label{辐射条件}
\end{equation}
\par 为了建立此问题的
积分方程,我们需要得到自由空间格林函数 $G_0$,它满足非齐次 Helmholtz 方程
\begin{equation}
    \nabla^2 G_0(\vec{\rho}, \vec{\rho}') + k_0^2 G_0(\vec{\rho}, \vec{\rho}') 
    = -\delta(\vec{\rho}-\vec{\rho}')
    \label{自由空间格林函数}
\end{equation}
\par 以及式 (\ref{辐射条件}) 的辐射条件。可以得到
\begin{equation}
    G_0(\vec{\rho}, \vec{\rho}') = 
    \frac{1}{4j}H_0^{(2)}
    \Big(
        k_0 \left|\vec{\rho}-\vec{\rho}'\right|
    \Big)
\end{equation}
\par 其中,$H_0^{(2)}$ 为零阶第二类 Hankel 函数。
\par 将式 (\ref{自由空间格林函数}) 乘以 $G_0$
并将式 (\ref{非齐次 Helmholtz 方程}) 乘以 $\varphi$,
然后对所得两式之差在整个外部区域上积分,得到
\begin{equation}
    \begin{aligned}
        \iint_{\Omega_{\infty}}&\Big[
            G_0(\vec{\rho}, \vec{\rho}') \nabla^2 \varphi(\vec{\rho})
            -\varphi(\vec{\rho}) \nabla^2 G_0(\vec{\rho}, \vec{\rho}')
        \Big]\text{d}\Omega\\
        &=\iint_{\Omega_{s}}
        G_0(\vec{\rho}, \vec{\rho}') f(\vec{\rho}) \text{d}\Omega
        +\iint_{\Omega_{\infty}}
        \varphi(\vec{\rho}) \delta(\vec{\rho}-\vec{\rho}') \text{d}\Omega
    \end{aligned}
\end{equation}
\par 其中,$\Omega_{s}$ 为源所在区域。应用第二标量 Green 公式
\begin{equation}
    \int_{\Omega} 
    (a\nabla^2 b-b\nabla^2 a)
    \text{d}\Omega
    =\oint_{\Gamma}
    \left(
        a \frac{\partial b}{\partial n}
        -b \frac{\partial a}{\partial n}
    \right)
    \text{d}\Gamma
\end{equation}
\par 其中,$\Gamma$ 为区域 $\Omega$ 的边界,可以得到
\begin{equation}
    \begin{aligned}
        \oint_{\Gamma_0}&\left[
            \varphi(\vec{\rho}) \frac{\partial G_0(\vec{\rho}, \vec{\rho}')}{\partial n}
            -G_0(\vec{\rho}, \vec{\rho'}) \frac{\partial \varphi(\vec{\rho})}{\partial n}
        \right]
        \text{d}\Gamma\\
        &+\oint_{\Gamma_{\infty}}\left[
            G_0(\vec{\rho}, \vec{\rho'}) \frac{\partial \varphi(\vec{\rho})}{\partial \rho}
            -\varphi(\vec{\rho}) \frac{\partial G_0(\vec{\rho}, \vec{\rho}')}{\partial \rho}
        \right]
        \text{d}\Gamma\\
        &-\iint_{\Omega_{s}}
        G_0(\vec{\rho}, \vec{\rho}') f(\vec{\rho}) \text{d}\Omega
        =
        \iint_{\Omega_{\infty}}
        \varphi(\vec{\rho}) \delta(\vec{\rho}-\vec{\rho}') \text{d}\Omega
    \end{aligned}
\end{equation}
\par 其中,$\Gamma_0$ 表示物体的边界,$\Gamma_{\infty}$ 表示无穷远处的边界。
\par 由于 $G_0$ 和 $\varphi$ 均满足辐射条件,故在 $\Gamma_{\infty}$ 上的边界积分项为零。
由 $\delta$ 函数的定义,可以得到
\begin{equation}
    \begin{aligned}
        \oint_{\Gamma_0}&\left[
            \varphi(\vec{\rho}) \frac{\partial G_0(\vec{\rho}, \vec{\rho}')}{\partial n}
            -G_0(\vec{\rho}, \vec{\rho'}) \frac{\partial \varphi(\vec{\rho})}{\partial n}
        \right]
        \text{d}\Gamma
        -\iint_{\Omega_{s}}
        G_0(\vec{\rho}, \vec{\rho}') f(\vec{\rho}) \text{d}\Omega\\
        &=\left\{
            \begin{aligned}
                \varphi(\vec{\rho}') \qquad &\vec{\rho}' \in \Omega_{\infty} \\
                0 \qquad \quad &\vec{\rho}' \in \Omega_{0}
            \end{aligned}
        \right.
    \end{aligned}
\end{equation}
\par 其中,$\Omega_0$ 为物体的内部区域。交换 $\vec{\rho}$ 
和 $\vec{\rho}'$,并利用 $G_0$ 的对称性,有
\begin{equation}
    \begin{aligned}
        \oint_{\Gamma_0}&\left[
            \varphi(\vec{\rho}') \frac{\partial G_0(\vec{\rho}, \vec{\rho}')}{\partial n'}
            -G_0(\vec{\rho}, \vec{\rho'}) \frac{\partial \varphi(\vec{\rho}')}{\partial n'}
        \right]
        \text{d}\Gamma'
        -\iint_{\Omega_{s}}
        G_0(\vec{\rho}, \vec{\rho}') f(\vec{\rho}') \text{d}\Omega'\\
        &=\left\{
            \begin{aligned}
                \varphi(\vec{\rho}) \qquad &\vec{\rho} \in \Omega_{\infty} \\
                0 \quad \quad \ \ \ &\vec{\rho} \in \Omega_{0}
            \end{aligned}
        \right.
    \end{aligned}
    \label{标量场 Huygens 原理}
\end{equation}

\begin{theorem}{标量场 Huygens 原理}
    在式 (\ref{标量场 Huygens 原理}) 中,
    当散射体不存在时,边界积分项为零。因此
    \begin{equation}
        \varphi(\rho) =
        -\iint_{\Omega_{s}}
        G_0(\vec{\rho}, \vec{\rho}') f(\vec{\rho}') \text{d}\Omega'
    \end{equation}
    此项称为入射波,记为 $\varphi^{\text{inc}}(\vec{\rho})$。
    式 (\ref{标量场 Huygens 原理}) 可以写为
    \begin{equation}
        \varphi^{\text{inc}}(\vec{\rho}) +
        \oint_{\Gamma_0}\left[
            \varphi(\vec{\rho}') \frac{\partial G_0(\vec{\rho}, \vec{\rho}')}{\partial n'}
            -G_0(\vec{\rho}, \vec{\rho'}) \frac{\partial \varphi(\vec{\rho}')}{\partial n'}
        \right]
        \text{d}\Gamma'
        =\left\{
            \begin{aligned}
                \varphi(\vec{\rho}) \qquad &\vec{\rho} \in \Omega_{\infty} \\
                0 \quad \quad \ \ \ &\vec{\rho} \in \Omega_{0}
            \end{aligned}
        \right.
        \label{标量场 Huygens 原理数学表示}
    \end{equation}
\end{theorem}

\par 为了从式 (\ref{标量场 Huygens 原理数学表示}) 建立积分
方程,我们需要将其应用于边界 $\Gamma_0$。
然而,当 $\vec{\rho}$ 位于边界 $\Gamma_0$ 上时,
$H_0^{(2)}\Big(k_0\left|\vec{\rho}-\vec{\rho}'\right|\Big)$ 及其导数在自变量为零时是奇异的。
\par 为了处理这一奇异性问题,
首先将 $\Gamma_0$ 分成两部分,一部分是 $\Gamma_0-2\epsilon$,
另一部分是圆心位于 $\vec{\rho}$ 且半径为 $\epsilon$ 的半圆,
然后令 $\epsilon\rightarrow 0$,这样就有
\begin{equation}
    \oint_{\Gamma_0}[\bullet]\text{d}\Gamma'
    =\lim_{\epsilon\rightarrow 0}
    \left\{
        \int_{\Gamma_0-2\epsilon}[\bullet]\text{d}\Gamma'
        +\int_{0}^{\pi}[\bullet]\epsilon\text{d}\phi
    \right\}
\end{equation}
\par 等号右边的第一个积分项
是一个沿 $\Gamma_0$ 但不包含奇异点的积分。对于第二个积分项,有
\begin{equation}
    \lim_{\epsilon\rightarrow 0}
    \int_{0}^{\pi}[\bullet]\epsilon\text{d}\phi
    =\frac{1}{4j}
    \lim_{\epsilon\rightarrow 0}
    \int_{0}^{\pi}\left[
        \varphi(\vec{\rho}') \frac{\partial H_0^{(2)}(k\epsilon)}{\partial \epsilon}
        -H_0^{(2)}(k\epsilon) \frac{\partial \varphi(\vec{\rho}')}{\partial n'}
    \right]\epsilon\text{d}\phi
\end{equation}
\par 由于 $\epsilon\rightarrow 0$ 时有
\begin{equation}
    H_0^{(2)}(z) \approx 
    1 - j\frac{2}{\pi}\ln\left(
        \frac{\gamma z}{2}
    \right) 
    \qquad
    \gamma = 1.7810724
\end{equation}
\par 可以得到
\begin{equation}
    \lim_{\epsilon\rightarrow 0}
    \int_{0}^{\pi}[\bullet]\epsilon\text{d}\phi
    =-\frac{1}{2}\varphi(\vec{\rho})
    \label{奇异积分项}
\end{equation}
\par 将式 (\ref{奇异积分项}) 代入式 (\ref{标量场 Huygens 原理数学表示}),
得到积分方程
\begin{equation}
    \varphi^{\text{inc}}(\vec{\rho}) +
    \oint_{\Gamma_0}\left[
        \varphi(\vec{\rho}') \frac{\partial G_0(\vec{\rho}, \vec{\rho}')}{\partial n'}
        -G_0(\vec{\rho}, \vec{\rho'}) \frac{\partial \varphi(\vec{\rho}')}{\partial n'}
    \right]
    \text{d}\Gamma'
    =\frac{1}{2}\varphi(\vec{\rho})
    \qquad
    \vec{\rho} \in \Gamma_0
    \label{标量场积分方程}
\end{equation}

\subsection{导电柱体的散射}

\subsection{导体条带的散射}

\subsection{均匀介质柱体的散射}

\section{三维分析}

\subsection{积分方程的建立}

\par 考虑自由空间中由电流密度为 $\vec{J}_{\text{imp}}$ 的源产生的被任意形状物体散射的电
磁场问题。其总场 $\vec{E}$ 和 $\vec{H}$ 满足矢量波动方程
\begin{gather}
    \label{矢量波动方程-1}
    \nabla \times \nabla \times \vec{E}(\vec{r}) - k_0^2 \vec{E}(\vec{r})
    = -jk_0Z_0 \vec{J}_{\text{imp}}(\vec{r}) \\
    \label{矢量波动方程-2}
    \nabla \times \nabla \times \vec{H}(\vec{r}) - k_0^2 \vec{H}(\vec{r})
    = \nabla \times \vec{J}_{\text{imp}}(\vec{r})
\end{gather}
\par 其中,$V_{\infty}$ 表示物体以外的区域。
同时,$\vec{E}$ 和 $\vec{H}$ 也满足在无穷远处
的 Sommerfeld 辐射条件
\begin{equation}
    r\left[
        \nabla \times
        \begin{pmatrix}
            \vec{E}\\
            \vec{H}
        \end{pmatrix}
        +jk_0\vec{r} \times
        \begin{pmatrix}
            \vec{E}\\
            \vec{H}
        \end{pmatrix}
    \right]=0
    \qquad
    r\rightarrow\infty
    \label{Sommerfeld 辐射条件}
\end{equation}
\par 为了推导 $\vec{E}$ 和 $\vec{H}$ 的积分方程,
我们引入三维空间中的 Green 函
数 $G_0$,其满足标量 Helmholtz 方程
\begin{equation}
    \label{三维空间中的 Green 函数方程}
    \nabla^2 G_0(\vec{r}, \vec{r}') + k_0^2 G_0(\vec{r}, \vec{r}') 
    = -\delta(\vec{r}-\vec{r}')
\end{equation}
\par 以及 Sommerfeld 辐射条件
\begin{equation}
    r \left[
        \frac{\partial G_0(\vec{r}, \vec{r}')}{\partial r}
        +j k_0 G_0(\vec{r}, \vec{r}')
    \right] = 0
    \qquad
    r \rightarrow \infty
    \label{Green 函数的 Sommerfeld 辐射条件}
\end{equation}
\par 其解为
\begin{equation}
    G_0(\vec{r}, \vec{r}') = 
    \frac{e^{-jk_0\left|\vec{r}-\vec{r}'\right|}}{4\pi\left|\vec{r}-\vec{r}'\right|}
\end{equation}
\par 接下来利用标量-矢量 Green 定理
\begin{equation}
    \begin{aligned}
        \iiint_V &\Big[
            b(\nabla\times\nabla\times a)
            +a\nabla^2 b
            +(\nabla \cdot a)\nabla b
        \Big]\text{d}V\\
        =
        &\oiint_S \Big[
            (\vec{n}\cdot a)\nabla b
            +(\vec{n}\times a)\times\nabla b
            +(\vec{n}\times\nabla\times a)b
        \Big]\text{d}S
    \end{aligned}
\end{equation}
\par 令 $V=V_{\infty}$,$\vec{a}=\vec{E}$,$b=G_0$,得到
\begin{equation}
    \label{标量-矢量 Green 定理应用}
    \begin{aligned}
        \iiint_{V_{\infty}} &\Big[
            G_0(\nabla\times\nabla\times \vec{E})
            +\vec{E}\nabla^2 G_0
            +(\nabla \cdot \vec{E})\nabla G_0
        \Big]\text{d}V\\
        =
        &\oiint_{S_{\infty}} \Big[
            (\vec{r}\cdot \vec{E})\nabla G_0
            +(\vec{r}\times \vec{E})\times\nabla G_0
            +(\vec{r}\times\nabla\times \vec{E})G_0
        \Big]\text{d}S\\
        &-
        \oiint_{S_0} \Big[
            (\vec{n}\cdot \vec{E})\nabla G_0
            +(\vec{n}\times \vec{E})\times\nabla G_0
            +(\vec{n}\times\nabla\times \vec{E})G_0
        \Big]\text{d}S\\
    \end{aligned}
\end{equation}
\par 使用式 (\ref{矢量波动方程-1}) 和式 (\ref{三维空间中的 Green 函数方程}),
式 (\ref{标量-矢量 Green 定理应用}) 的左边变为
\begin{equation}
    \begin{aligned}
        \text{LHS}
        =\iiint_{V_{\infty}} &\left[
            -jk_0Z_0 \vec{J}_{\text{imp}} G_0 
            -\vec{E}\delta(\vec{r}-\vec{r}')
            +\frac{jZ_0}{k_0}(\nabla \cdot \vec{J}_{\text{imp}})
            \nabla G_0
        \right]\text{d}V\\
        =
        &-jk_0Z_0\iiint_{V_{s}} \left[
            \vec{J}_{\text{imp}} G_0
            -\frac{1}{k_0^2}(\nabla \cdot \vec{J}_{\text{imp}})
            \nabla G_0
        \right]\text{d}V
        -
        \left\{
            \begin{aligned}
                \vec{E}(\vec{r}')  \qquad &\vec{r}' \in V_{\infty} \\
                \vec{0} \ \ \ \qquad &\vec{r}' \in V_{0}
            \end{aligned}
        \right.
    \end{aligned}
\end{equation}
\par 其中,$V_{s}$ 为源所在区域。
使用式 (\ref{Sommerfeld 辐射条件}) 和式 (\ref{Green 函数的 Sommerfeld 辐射条件}),
可以证明式 (\ref{标量-矢量 Green 定理应用}) 的右边第一项为零。
因此式 (\ref{标量-矢量 Green 定理应用}) 变为
\begin{equation}
    \begin{aligned}
        \oiint_{S_0} &\Big[
            (\vec{n}\cdot \vec{E})\nabla G_0
            +(\vec{n}\times \vec{E})\times\nabla G_0
            +(\vec{n}\times\nabla\times \vec{E})G_0
        \Big]\text{d}S\\
        &-jk_0Z_0\iiint_{V_{s}} \left[
            \vec{J}_{\text{imp}} G_0
            -\frac{1}{k_0^2}(\nabla \cdot \vec{J}_{\text{imp}})
            \nabla G_0
        \right]\text{d}V
        =\left\{
            \begin{aligned}
                \vec{E}(\vec{r}') \qquad &\vec{r}' \in V_{\infty} \\
                \vec{0} \ \ \ \qquad &\vec{r}' \in V_{0}
            \end{aligned}
        \right.
    \end{aligned}
    \label{三维积分方程}
\end{equation}
\par 交换带撇和不带撇的坐标后,上式可以写为
\begin{equation}
    \begin{aligned}
        \oiint_{S_0} &\Big[
            (\vec{n}'\cdot \vec{E})\nabla' G_0
            +(\vec{n}'\times \vec{E})\times\nabla' G_0
            -jk_0Z_0(\vec{n}'\times\vec{H})G_0
        \Big]\text{d}S'\\
        &-jk_0Z_0\iiint_{V_{s}} \left[
            \vec{J}_{\text{imp}} G_0
            -\frac{1}{k_0^2}(\nabla' \cdot \vec{J}_{\text{imp}})
            \nabla' G_0
        \right]\text{d}V'
        =\left\{
            \begin{aligned}
                \vec{E}(\vec{r}) \qquad &\vec{r} \in V_{\infty} \\
                \vec{0} \ \ \ \qquad &\vec{r} \in V_{0}
            \end{aligned}
        \right.
    \end{aligned}
    \label{三维积分方程变式}
\end{equation}

\begin{theorem}{三维电场积分方程}
    在式 (\ref{三维积分方程变式}) 中,
    当散射体不存在时,边界积分项为零。因此
    \begin{equation}
        \vec{E}(\vec{r}) =
        -jk_0Z_0\iiint_{V_{s}} \left[
            \vec{J}_{\text{imp}} G_0
            -\frac{1}{k_0^2}(\nabla' \cdot \vec{J}_{\text{imp}})
            \nabla' G_0
        \right]\text{d}V'
    \end{equation}
    此项称为入射场,记为 $\vec{E}^{\text{inc}}(\vec{r})$。
    式 (\ref{三维积分方程变式}) 可以写为
    \begin{equation}
        \begin{aligned}
            \vec{E}^{\text{inc}}(\vec{r})
            &-\oiint_{S_0} \Big[
                (\vec{n}'\cdot \vec{E})\nabla G_0
                +(\vec{n}'\times \vec{E})\times\nabla G_0
                +jk_0Z_0(\vec{n}'\times\vec{H})G_0
            \Big]\text{d}S'\\
            &=\left\{
                \begin{aligned}
                    \vec{E}(\vec{r}) \qquad &\vec{r} \in V_{\infty} \\
                    \vec{0} \ \ \ \qquad &\vec{r} \in V_{0}
                \end{aligned}
            \right.
            \label{三维电场积分方程-1}
        \end{aligned}
    \end{equation}
    在得到上式的过程中,使用了
    $\nabla'G_0=-\nabla G_0$。

    根据面矢量分析,可以证明
    \begin{equation}
        \vec{n}'\cdot\vec{E}
        =\frac{jZ_0}{k_0}\nabla'\cdot(\vec{n}'\times\vec{H})
    \end{equation}
    式 (\ref{三维电场积分方程-1}) 可以写为
    \begin{equation}
        \begin{aligned}
            \vec{E}^{\text{inc}}(\vec{r})
            &-\oiint_{S_0} \left[
                \frac{jZ_0}{k_0}\nabla'\cdot(\vec{n}'\times\vec{H})\nabla G_0
                +(\vec{n}'\times \vec{E})\times\nabla G_0
                +jk_0Z_0(\vec{n}'\times\vec{H})G_0
            \right]\text{d}S'\\
            &=\left\{
                \begin{aligned}
                    \vec{E}(\vec{r}) \qquad &\vec{r} \in V_{\infty} \\
                    \vec{0} \ \ \ \qquad &\vec{r} \in V_{0}
                \end{aligned}
            \right.
        \end{aligned}
        \label{三维电场积分方程-2}
    \end{equation}
\end{theorem}

\begin{theorem}{三维磁场积分方程}
    类似地,可以得到
    \begin{equation}
        \begin{aligned}
            \vec{H}^{\text{inc}}(\vec{r})
            &-\oiint_{S_0} \left[
                \frac{Y_0}{jk_0}\nabla'\cdot(\vec{n}'\times\vec{E})\nabla G_0
                +(\vec{n}'\times \vec{H})\times\nabla G_0
                -jk_0Y_0(\vec{n}'\times\vec{E})G_0
            \right]\text{d}S'\\
            &=\left\{
                \begin{aligned}
                    \vec{H}(\vec{r}) \qquad &\vec{r} \in V_{\infty} \\
                    \vec{0} \ \ \ \qquad &\vec{r} \in V_{0}
                \end{aligned}
            \right.
        \end{aligned}
        \label{三维磁场积分方程}
    \end{equation}
    其中
    \begin{equation}
        \vec{H}^{\text{inc}}(\vec{r})
        =\iiint_{V_s}\vec{J}_{\text{imp}}\times\nabla' G_0\text{d}V'
    \end{equation}
\end{theorem}

\begin{definition}
    定义算子
    \begin{align}
        \label{矩量法算子定义-1}
        \mathcal{L}(\vec{X})
        &=jk_0\oiint_{S_0} \left[
            \vec{X}(\vec{r}')G_0(\vec{r}, \vec{r}')
            +\frac{1}{k_0^2}\nabla' \cdot \vec{X}(\vec{r}')
            \nabla' G_0(\vec{r}, \vec{r}')
        \right]\text{d}S'\\
        \label{矩量法算子定义-2}
        \mathcal{K}(\vec{X})
        &=\oiint_{S_0}
            \vec{X}(\vec{r}')\times\nabla' G_0(\vec{r}, \vec{r}')
        \text{d}S'
    \end{align}
    并引入等效面电流和面磁流
    \begin{equation}
        \vec{J}_s(\vec{r}')
        =\vec{n}'\times\vec{H}(\vec{r}')
        \qquad
        \vec{M}_s(\vec{r}')
        =\vec{E}(\vec{r}')\times\vec{n}'
    \end{equation}
\end{definition}

\par 式 (\ref{三维电场积分方程-2}) 和式 (\ref{三维磁场积分方程}) 可以写为
\begin{equation}
    \vec{E}^{\text{inc}}(\vec{r})-\mathcal{L}(\vec{\bar{J}}_s)
    +\mathcal{K}(\vec{M}_s)=
    \left\{
        \begin{aligned}
            \vec{E}(\vec{r}) \qquad &\vec{r} \in V_{\infty} \\
            \vec{0} \ \ \ \qquad &\vec{r} \in V_{0}
        \end{aligned}
    \right.
    \label{表面电场积分方程}
\end{equation}
\begin{equation}
    \vec{\bar{H}}^{\text{inc}}(\vec{r})-\mathcal{L}(\vec{M}_s)
    -\mathcal{K}(\vec{\bar{J}}_s)=
    \left\{
        \begin{aligned}
            \vec{\bar{H}}(\vec{r}) \qquad &\vec{r} \in V_{\infty} \\
            \vec{0} \ \ \ \qquad &\vec{r} \in V_{0}
        \end{aligned}
    \right.
    \label{表面磁场积分方程}
\end{equation}
\par 为了使方程形式更简洁,
我们按比例缩放了电流密度 $(\vec{\bar{J}}_s=Z_0\vec{J}_s)$ 
和磁场强度 $(\vec{\bar{H}}=Z_0\vec{H})$。
\par 我们可以将这两个式子
叉乘 $\vec{n}$,然后令 $\vec{r}$ 趋近于 $S_0$,
以此建立积分方程来求解 $\vec{\bar{J}}_s$ 和
$\vec{M}_s$。同二维情况一样,我们需要处理在表面上时
$\mathcal{L}(\vec{X})$ 和 $\mathcal{K}(\vec{X})$ 的奇异性。
\par 将面 $S_0$ 分成两部分:
一部分是 $S_0$ 减去以 $\vec{r}$ 为中心的
一个无限小圆片,另一部分是半径为 $\epsilon$ 的无限小半
球面。因此
\begin{align}
    \mathcal{L}(\vec{X})
    &=\lim_{\epsilon\rightarrow 0}
    \left\{
        \iint_{S_0-\sigma_{\epsilon}}
        [\bullet]\text{d}S'
        +\int_{0}^{2\pi}
        \int_{0}^{\frac{\pi}{2}}
        [\bullet]\epsilon^2 \sin\theta \text{d}\theta\text{d}\phi
    \right\}\\
    \mathcal{K}(\vec{X})
    &=\lim_{\epsilon\rightarrow 0}
    \left\{
        \iint_{S_0-\sigma_{\epsilon}}
        [\bullet]\text{d}S'
        +\int_{0}^{2\pi}
        \int_{0}^{\frac{\pi}{2}}
        [\bullet]\epsilon^2 \sin\theta \text{d}\theta\text{d}\phi
    \right\}
\end{align}
\par 奇异项对计算 $\vec{n}\times \mathcal{L}(\vec{X})$ 没有贡献,
而在 $\vec{n}\times \mathcal{K}(\vec{X})$ 中则不为零。
为了计算此项,令
$\vec{X}=\vec{n}\times \vec{Y}$,并考虑
\begin{equation}
    \begin{aligned}
        \vec{n}\times \int_{0}^{2\pi}
        \int_{0}^{\frac{\pi}{2}}
        \vec{X}\times\nabla G_0 \epsilon^2 \sin\theta \text{d}\theta\text{d}\phi
        &=
        \pm \vec{n}\times \int_{0}^{2\pi}
        \int_{0}^{\frac{\pi}{2}}
        (\vec{n}\times\vec{Y})\times \vec{r}' 
        \frac{e^{-jk_0\epsilon}}{4\pi}
        \sin \theta \text{d}\theta\text{d}\phi\\
        &=
        \pm \vec{n}\times \int_{0}^{2\pi}
        \int_{0}^{\frac{\pi}{2}}
        (\vec{n}'\cdot\vec{r}')\vec{Y} \frac{e^{-jk_0\epsilon}}{4\pi}
        \sin \theta \text{d}\theta\text{d}\phi\\
        &=
        \pm \vec{n}\times 
        \int_{0}^{2\pi} \int_{0}^{\frac{\pi}{2}}
        \vec{Y} \frac{e^{-jk_0\epsilon}}{4\pi}
        \sin \theta \text{d}\theta\text{d}\phi\\
        &=\pm \frac{1}{2} \vec{X}
        \qquad
        \epsilon\rightarrow 0
    \end{aligned}
\end{equation}
\par 其中,$\vec{r}'$ 是半球面的单位法向矢量。
因此,$\vec{n}\times\mathcal{K}(\vec{X})$ 可写为
\begin{equation}
    \vec{n}\times \mathcal{K}(\vec{X})
    =\vec{n}\times \tilde{\mathcal{K}}(\vec{X})
    \pm \frac{1}{2}\vec{X}
    \label{矩量法奇异积分}
\end{equation}
\par 式中的 $\tilde{\mathcal{K}}(\vec{X})$ 与
式 (\ref{矩量法算子定义-2}) 中的积分相同,
只是去除了奇异点 $\vec{r}=\vec{r}'$。
\par 式 (\ref{表面电场积分方程}) 和
式 (\ref{表面磁场积分方程}) 与 $\vec{n}$ 叉乘,
并利用式 (\ref{矩量法奇异积分}),得到
\begin{align}
    \frac{1}{2}\vec{M}_s
    -\vec{n}\times\mathcal{L}(\vec{\bar{J}}_s)
    +\vec{n}\times\tilde{\mathcal{K}}(\vec{M}_s)
    &=-\vec{n}\times\vec{E}^{\text{inc}}
    \qquad
    \vec{r}\in S_0\\
    \frac{1}{2}\vec{\bar{J}}_s
    +\vec{n}\times\mathcal{L}(\vec{M}_s)
    +\vec{n}\times\tilde{\mathcal{K}}(\vec{\bar{J}}_s)
    &=\vec{n}\times\vec{\bar{H}}^{\text{inc}}
    \qquad \ \ 
    \vec{r}\in S_0
\end{align}

\section{时域矩量法}

\subsection{时域积分方程}

\par 时域矩量法分析首先需要建立时域积分方程。
这可以通过对频域积分方程进行 Fourier 逆变换来得到。
考虑一个导电体的散射问
题,它的电场积分方程可以写为
\begin{equation}
    \vec{n}\times
    \oiint_{S_0} \Big[
        (jk_0)^2\vec{\bar{J}}_s(\vec{r}')G_0(\vec{r}, \vec{r}')
        -\nabla'\cdot\vec{\bar{J}}_s(\vec{r}')\nabla G_0(\vec{r}, \vec{r}')
    \Big]\text{d}S'
    =jk_0\vec{n}\times\vec{E}^{\text{inc}}(\vec{r})
\end{equation}
\par 进行 Fourier 逆变换得到
\begin{equation}
    \vec{n}\times
    \oiint_{S_0} \left[
        \frac{\partial^2}{\partial t^2}    
        \frac{\vec{\bar{\mathscr{J}}}_s\left(\vec{r}',t-\frac{R}{c_0}\right)}{4\pi c_0R}
        -c_0\nabla 
        \frac{\nabla' \cdot
        \vec{\bar{\mathscr{J}}}_s\left(\vec{r}',t-\frac{R}{c_0}\right)}{4\pi R}
    \right]\text{d}S'
    =\vec{n}\times
    \frac{\partial \mathscr{E}^{\text{inc}}(\vec{r},t)}{\partial t}
\end{equation}
\par 类似地,时域磁场积分方程表示为
\begin{equation}
    \frac{1}{2}
    \frac{\partial \vec{\bar{\mathscr{J}}}_s^{\text{inc}}(\vec{r},t)}{\partial t}
    -\vec{n}\times
    \oiint_{S_0} \nabla \times
    \frac{\partial}{\partial t}
    \frac{\vec{\bar{\mathscr{J}}}_s\left(\vec{r}',t-\frac{R}{c_0}\right)}{4\pi c_0R}
    \text{d}S'
    =\vec{n}\times
    \vec{\bar{\mathscr{H}}}_s^{\text{inc}}(\vec{r},t)
\end{equation}

\subsection{时间步进求解}

\par 为了求解时域积分方程,要将未知函数在空间域和时域上展开。
首先进行空间展开,则 $\vec{\bar{\mathscr{J}}}_s\left(\vec{r}',t\right)$ 可表示为
\begin{equation}
    \vec{\bar{\mathscr{J}}}_s\left(\vec{r}',t\right)
    \approx\sum_{n=1}^{N_s} \mathscr{J}_n(t) \vec{\Lambda}_n(\vec{r}')
\end{equation}
\par 其中,$N_s$ 为三角形网格中的公共边数目。
然后,我们还需要对其进行时域上的展开
\begin{equation}
    \mathscr{J}_n(t)
    \approx\sum_{l=1}^{N_t} 
    I_n^{(l)}(t)T_l(t)
\end{equation}
\par 其中,$N_t$ 为时间步的总数目,
$T_l(t)=T(t-l\Delta t)$ 为时域基函数。
\par 由此可以得到未知函数的完整离散形式为
\begin{equation}
    \vec{\bar{\mathscr{J}}}_s\left(\vec{r}',t\right)
    \approx\sum_{n=1}^{N_s} 
    \sum_{l=1}^{N_t} 
    I_n^{(l)} \vec{\Lambda}_n(\vec{r}')T_l(t)
\end{equation}

\begin{theorem}{时间步进公式}
    空间域测试函数选择 $\Lambda_m(\vec{r})$,
    时域测试函数选择 $\delta(t-k\Delta t)$,得到
    \begin{equation}
        [Z^{(0)}]\{I\}^{(k)}
        =\{V\}^{(k)}-\sum_{l=1}^{k-1}[Z^{(k-l)}]\{I\}^{(l)}
        \qquad 
        k=1,2,\cdots
    \end{equation}
    其中,$\{I\}^{(k)}
    =[I_1^{(k)},I_2^{(k)},\cdots,I_{N_s}^{(k)}]^T$,
    此外
    \begin{align}
        Z_{mn}^{(k-l)}
        &=\oiint_{S_0}
        \vec{\Lambda}_m(\vec{r})\cdot
        \oiint_{S_0}
        \left[
            \frac{\vec{\Lambda}_n(\vec{r}')}{4\pi c_0 R}
            \ddot{T}^{(k-l)}-c_0\nabla 
            \frac{\nabla' \cdot \vec{\Lambda}_n(\vec{r}')}{4\pi R}T^{(k-l)}
        \right]\text{d}S'\text{d}S\\
        V_m^{(k)}
        &=\oiint_{S_0}
        \vec{\Lambda}_m(\vec{r})\cdot
        \vec{\dot{\mathscr{E}}}^{\text{inc}}(\vec{r},k\Delta t)\text{d}S
    \end{align}
    其中,$T^{(k-l)}=
    T\left((k-l)\Delta t - \frac{R}{c_0}\right)$,
    $\ddot{T}=\frac{\partial^2 T(t)}{\partial t^2}$,
    $\vec{\dot{\mathscr{E}}}^{\text{inc}}
    =\frac{\partial \vec{\mathscr{E}}^{\text{inc}}(\vec{r},t)}{\partial t}$。
\end{theorem}